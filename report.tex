\documentclass{article}
\usepackage{fancyhdr}
\usepackage{minted}
\usepackage[margin=1in]{geometry}

\pagestyle{fancy}
\lhead{Database System Report}
\rhead{CS 5330 Fall 2023}
\renewcommand{\headrulewidth}{0pt}

\begin{document}
	\section*{Database Schema Overview}
		\begin{enumerate}
		    \item \textbf{Table:} \texttt{departments}
		    \begin{itemize}
		        \item \textbf{Purpose:} Represents university departments.
		        \item \textbf{Fields:}
		        \begin{itemize}
		            \item \texttt{id}: Primary key, uniquely identifies a department.
		            \item \texttt{name}: Name of the department (unique).
		            \item \texttt{code}: A unique, maximum four-character code representing the department.
		        \end{itemize}
		        \item \textbf{Relationships:} Has many \texttt{Faculty} members and \texttt{Programs}.
		    \end{itemize}

		    \item \textbf{Table:} \texttt{faculty}
		    \begin{itemize}
		        \item \textbf{Purpose:} Stores information about faculty members.
		        \item \textbf{Fields:}
		        \begin{itemize}
		            \item \texttt{id}: Primary key, uniquely identifies a faculty member.
		            \item \texttt{name}: Name of the faculty member.
		            \item \texttt{email}: Email address of the faculty member (unique).
		            \item \texttt{rank}: Academic rank (full, associate, assistant, adjunct).
		            \item \texttt{department\_code}: Foreign key linking to the \texttt{departments} table.
		        \end{itemize}
		        \item \textbf{Relationships:} Belongs to a \texttt{Department}.
		    \end{itemize}

		    \item \textbf{Table:} \texttt{programs}
		    \begin{itemize}
		        \item \textbf{Purpose:} Represents academic programs or degrees.
		        \item \textbf{Fields:}
		        \begin{itemize}
		            \item \texttt{id}: Primary key, uniquely identifies a program.
		            \item \texttt{name}: Name of the program (unique).
		            \item \texttt{department\_code}: Foreign key linking to the \texttt{departments} table.
		            \item \texttt{in\_charge\_id}: Foreign key linking to a \texttt{Faculty} member who is in charge of the program.
		        \end{itemize}
		        \item \textbf{Relationships:} Belongs to a \texttt{Department} and has many \texttt{Courses} through \texttt{program\_courses}.
		    \end{itemize}

		    \item \textbf{Table:} \texttt{courses}
		    \begin{itemize}
		        \item \textbf{Purpose:} Contains information about courses offered by departments.
		        \item \textbf{Fields:}
		        \begin{itemize}
		            \item \texttt{id}: Primary key, a combination of department code and a 4-digit number.
		            \item \texttt{title}: Title of the course.
		            \item \texttt{description}: A text describing the course.
		            \item \texttt{department\_code}: Foreign key linking to the \texttt{departments} table.
		        \end{itemize}
		        \item \textbf{Relationships:} Belongs to a \texttt{Department}, has many \texttt{Sections}, and is associated with many \texttt{Programs} through \texttt{program\_courses}.
		    \end{itemize}

		    \item \textbf{Table:} \texttt{sections}
		    \begin{itemize}
		        \item \textbf{Purpose:} Details about specific course offerings each semester.
		        \item \textbf{Fields:}
		        \begin{itemize}
		            \item \texttt{id}: Primary key, uniquely identifies a section.
		            \item \texttt{number}: A 3-digit number representing the section.
		            \item \texttt{semester}: The semester when the course is offered.
		            \item \texttt{year}: The year when the course is offered.
		            \item \texttt{course\_id}: Foreign key linking to the \texttt{courses} table.
		            \item \texttt{instructor\_id}: Foreign key linking to the \texttt{faculty} table.
		            \item \texttt{enrollment\_count}: Number of students enrolled in the section.
		        \end{itemize}
		        \item \textbf{Relationships:} Belongs to a \texttt{Course} and has a \texttt{Faculty} member as the instructor.
		    \end{itemize}

		    \item \textbf{Table:} \texttt{learning\_objectives}
		    \begin{itemize}
		        \item \textbf{Purpose:} Stores learning objectives for program evaluation.
		        \item \textbf{Fields:}
		        \begin{itemize}
		            \item \texttt{id}: Primary key, a code for the learning objective.
		            \item \texttt{description}: Description of the learning objective.
		            \item \texttt{parent\_id}: Foreign key to itself, indicating sub-objectives.
		        \end{itemize}
		        \item \textbf{Relationships:} Has many sub-objectives linking to itself.
		    \end{itemize}

		    \item \textbf{Table:} \texttt{program\_courses}
		    \begin{itemize}
		        \item \textbf{Purpose:} A junction table to establish many-to-many relationships between programs and courses.
		        \item \textbf{Fields:}
		        \begin{itemize}
		            \item \texttt{program\_id}: Foreign key linking to the \texttt{programs} table.
		            \item \texttt{course\_id}: Foreign key linking to the \texttt{courses} table.
		        \end{itemize}
		        \item \textbf{Relationships:} Links \texttt{Programs} to \texttt{Courses}.
		    \end{itemize}

		    \item \textbf{Table:} \texttt{course\_objectives}
		    \begin{itemize}
		        \item \textbf{Purpose:} Associates courses with their respective learning objectives.
		        \item \textbf{Fields:}
		        \begin{itemize}
		            \item \texttt{course\_id}: Foreign key linking to the \texttt{courses} table.
		            \item \texttt{objective\_id}: Foreign key linking to the \texttt{learning\_objectives} table.
		            \item \texttt{program\_id}: Foreign key linking to the \texttt{programs} table.
		        \end{itemize}
		        \item \textbf{Relationships:} Links \texttt{Courses} to \texttt{Learning Objectives}.
		    \end{itemize}

		    \item \textbf{Table:} \texttt{section\_evaluations}
		    \begin{itemize}
		        \item \textbf{Purpose:} Records the evaluation of learning objectives for each course section.
		        \item \textbf{Fields:}
		        \begin{itemize}
		            \item \texttt{section\_id}: Foreign key linking to the \texttt{sections} table.
		            \item \texttt{objective\_id}: Foreign key linking to the \texttt{learning\_objectives} table.
		            \item \texttt{evaluation\_method}: The method used for evaluating the objective.
		            \item \texttt{students\_met}: Number of students who met the objective.
		        \end{itemize}
		        \item \textbf{Relationships:} Links \texttt{Sections} to \texttt{Learning Objectives}.
		    \end{itemize}
		\end{enumerate}

		\section*{Installation}
		Before installing the University Program Evaluation System, ensure that your environment meets the following prerequisites.

		\subsection{Prerequisites}
		\begin{itemize}
		    \item Python 3.8 or later
		    \item Pip (Python package installer)
		    \item A database system supported by SQLAlchemy (e.g., SQLite, PostgreSQL, MySQL)
		    \item Tkinter for the graphical user interface (usually comes pre-installed with Python)
		\end{itemize}

		\subsection{Setting up the Environment}
		\begin{enumerate}
		    \item Install Python and Pip and follow the instructions for your operating system.
		    \item Install required Python libraries:
		    \begin{minted}{bash}
pip install sqlalchemy tk
		    \end{minted}
		    \item Install a database system of your choice. This guide will use SQLite for simplicity.
		    \begin{minted}{bash}
pip install sqlite
		    \end{minted}
		\end{enumerate}

		\subsection{Downloading and Configuring the Application}
		\begin{enumerate}
		    \item Download the application source code from the provided repository or location.
		    \item Extract the contents if in a compressed format.
		    \item Navigate to the application's root directory in your terminal.
		    \item Modify the `main.py` file to set up the database connection. Here's an example using SQLite:
		    \begin{minted}{python}
# main.py
from sqlalchemy import create_engine
engine = create_engine('sqlite:///university_programs.db')
		    \end{minted}
		\end{enumerate}

		\section*{Usage}
		This section covers how to run and use the University Program Evaluation System.

		\subsection{Starting the Application}
		\begin{enumerate}
		    \item Open a terminal.
		    \item Navigate to the application's root directory.
		    \item Run the following command:
		    \begin{minted}{bash}
python gui.py
		    \end{minted}
		    \item This will open the graphical user interface of the application.
		\end{enumerate}

		\subsection{Navigating the Interface}
		The application has various tabs for data entry, querying, and settings. Each tab is dedicated to a specific function.

		\subsubsection{Data Entry}
		You can enter data for departments, faculty, programs, courses, etc. To enter data:
		\begin{enumerate}
		    \item Click on the ‘Data Entry’ tab.
		    \item Choose the category (e.g., Departments, Faculty) you want to add data to.
		    \item Fill in the required fields.
		    \item Click ‘Submit’ to save the data.
		\end{enumerate}

		\subsubsection{Data Query}
		To query data:
		\begin{enumerate}
		    \item Click on the ‘Data Query’ tab.
		    \item Choose the category you want to query (e.g., Department, Program).
		    \item Fill in the query parameters.
		    \item Click ‘Submit’ to view the results.
		\end{enumerate}

		\subsubsection{Settings}
		To reset the database:
		\begin{enumerate}
		    \item Click on the ‘Settings’ menu in the top bar.
		    \item Select ‘Reset Database’.
		    \item Confirm the action if prompted.
		\end{enumerate}

		\section*{Implementation Manual}
			\subsection*{gui.py}
			\begin{enumerate}
			    \item \textbf{Validation Functions}: A collection of functions for validating user input in GUI forms. Each function corresponds to a specific type of input, ensuring correctness and consistency of data.
			    \item \textbf{Data Handling Functions}: These functions interact with the database to perform operations like adding, updating, and querying data based on user inputs from the GUI.
			    \item \textbf{GUI Setup Functions}: Responsible for initializing and configuring various components of the GUI, such as tabs, buttons, entry fields, etc.
			    \item \textbf{Entry and Query Field Setup}: Functions that create and manage entry fields for data input and query operations in the GUI.
			    \item \textbf{Main GUI Initialization}: The entry point for initializing the entire GUI application.
			\end{enumerate}

			\subsection*{db.py}
			\begin{enumerate}
			    \item \textbf{SessionManager Class}: Manages database sessions, providing an interface for executing various database operations.
			    \item \textbf{Database Operation Functions}: Functions for adding and querying different entities in the database such as departments, faculty, programs, courses, etc.
			    \item \textbf{Specialized Query Functions}: Includes complex queries for retrieving specific data, like program evaluation results based on semester or year.
			\end{enumerate}

			\subsection*{main.py}
			\begin{enumerate}
			    \item \textbf{Database Initialization}: Establishes the initial connection to the database and sets up the database schema.
			    \item \textbf{GUI Launch}: The starting point of the application, responsible for launching the GUI.
			\end{enumerate}

			\section*{Program Code}
			\subsection*{main.py}
			\inputminted[fontsize=\footnotesize, linenos]{python}{main.py}
			\newpage

			\subsection*{gui.py}
			\inputminted[fontsize=\footnotesize, linenos]{python}{gui.py}
			\newpage

			\subsection*{db.py}
			\inputminted[fontsize=\footnotesize, linenos]{python}{db.py}
			

\end{document}
